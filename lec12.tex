\documentclass{jsarticle}
\title{講義ノート第12回}
\author{}
\date{}


\usepackage[top=30truemm,bottom=30truemm,left=25truemm,right=25truemm]{geometry}

\usepackage{ascmac}

\usepackage{amsmath}

\usepackage[dvips]{graphicx}

\usepackage{ulem}

\parindent = 0pt

\begin{document}
\maketitle

\setcounter{section}{2}

\section{電磁誘導}

\setcounter{subsection}{1}

\subsection{自己誘導,相互誘導}
定電圧Vの直流電源,抵抗値Rの抵抗,インダクタンスLのコイルからなる直列回路を考える.
\begin{eqnarray}
V=RI+L\frac{dI}{dt}
\end{eqnarray}

\begin{eqnarray*}
\frac{dI}{dI} = - \frac{R}{L}I + \frac{V}{L} = -\frac{R}{L} \left( I - \frac{V}{R} \right)
\end{eqnarray*}

ここでVは定電圧であるから
\begin{eqnarray*}
\frac{d}{dt} \left( I - \frac{V}{R} \right) =  -\frac{R}{L} \left( I - \frac{V}{R} \right) \\
I - \frac{V}{R} = k e^{-\frac{R}{L}t}
\end{eqnarray*}
$t=0$で$I=0$ であるとすると, $k=-\frac{V}{R}$
\begin{eqnarray*}
I = \frac{V}{R} - \frac{V}{R} e^{-\frac{R}{L}t} = \frac{V}{R} \left( 1 - e^{-\frac{R}{L}t} \right)
\end{eqnarray*}
式から分かるように,定常状態に落ち着くまで電流は時間的に変化する,このような現象を{\bf 過渡現象}という.
式(1)の両辺に電流Iを掛けると
\begin{eqnarray*}
IV = RI^2 + LI \frac{dI}{dt} = RI^2 + \frac{d}{dt} \left( \frac{1}{2} LI^2 \right)
\end{eqnarray*}
$IV$は電源の仕事率,$RI^2$ は単位時間当たりのジュール熱, $\frac{1}{2} LI^2$ は誘導起電力に抗して電流を0からIにするのに要するエネルギー,を表す. \\
さて,ソレノイドのエネルギー $\frac{1}{2} LI^2=\frac{1}{2} \Phi I$を,磁束密度$B=\mu_0 n I$,磁束$\Phi = lnBS$を用いて書き換える.
\begin{eqnarray*}
\frac{1}{2} \Phi I &=& \frac{1}{2} lnBSI = \frac{1}{2} lS \cdot \frac{B}{\mu_0} B \\
&=& lS \cdot \frac{1}{2\mu_0}B^2
\end{eqnarray*}
$lS$はソレノイドの体積であり,$\frac{1}{2} \Phi I$はエネルギーであるから,$\frac{1}{2\mu_0}B^2$はエネルギーの密度を表していることになる. \\
一般に磁場{\bf B} は{\bf エネルギー密度}$\frac{1}{2\mu_0}\left| {\bf B} \right|^2$ をもつ. \\
電流の系について,エネルギーを場による表現に書き換える.
\begin{eqnarray*}
\frac{1}{2} \sum_{k} \Phi_{k} I_{k} = \int \frac{1}{2 \mu_0} \left| {\bf B} \right|^2 dV
\end{eqnarray*}
また電荷の系についても,エネルギーを場による表現に書き換えることができるのは以前示した通りで,再掲すると \\
\begin{eqnarray*}
\frac{1}{2} \sum_{k} q_k \phi_k = \frac{\varepsilon_0}{2} \int \left| {\bf E} \right|^2 dV
\end{eqnarray*}

\subsection{変位電流とAmp\`ere-Maxwellの法則}
Amp\`ereの法則は
\begin{eqnarray*}
{\rm rot}{\bf B} = \mu_0 {\bf i}
\end{eqnarray*}
両辺の発散をとると,
\begin{eqnarray*}
{\rm div ( rot {\bf B}) } = \mu_0 {\rm div}{\bf i}
\end{eqnarray*}
\begin{eqnarray*}
\mbox{左辺}= \nabla \cdot ( \nabla \times {\bf B} ) = 0
\end{eqnarray*}
連続の式${\rm div}{\bf i}=-\frac{\partial \rho}{\partial t}$ を用いると右辺は
\begin{eqnarray*}
\mbox{右辺}= \mu_0 \left( -\frac{\partial \rho}{\partial t} \right)
\end{eqnarray*}
となるから,電荷密度に時間変化がある場合は,Amp\`ereの法則がこのままの形では成り立たない. \\
補正形として何らかのベクトル場{\bf X}を付け加えることにより,${\rm rot}{\bf B} = \mu_0 ( {\bf i} + {\bf X} )$ を仮定する. \\
両辺の発散をとると,
\begin{eqnarray*}
0 = \mu_0 ( {\rm div}{\bf i} + {\rm div}{\bf X} ) = \mu_0 \left( -\frac{\partial \rho}{\partial t} + {\rm div}{\bf X} \right)
\end{eqnarray*}
Gaussの法則${\rm div}{\bf E}=\frac{\rho}{\varepsilon_0}$ を用いて,
\begin{eqnarray*}
0 &=& \mu_0 \left( -\frac{\partial}{\partial t} \varepsilon_0 {\rm div}{\bf E} + {\rm div}{\bf X} \right) \\
&=& \mu_0 \left(  {\rm div} \frac{\partial {\bf E}}{\partial t}   + {\rm div}{\bf X} \right) \\
&=& \mu_0 {\rm div} \left( - \varepsilon_0 \frac{\partial {\bf E}}{\partial t} + {\bf X} \right)
\end{eqnarray*}
これが整合するためには,${\bf X}=\varepsilon_0 \frac{\partial {\bf E}}{\partial t}$ ととればよい.
$\varepsilon_0 \frac{\partial {\bf E}}{\partial t}$のことを,{\bf 変位電流密度}あるいは{\bf 電束電流密度}という.
\begin{itembox}[c]{Amp\`ere-Maxwellの法則}
\begin{eqnarray}
{\rm rot}{\bf B}=\mu_0 \left( {\bf i} + \varepsilon_0 \frac{\partial {\bf E}}{\partial t} \right)
\end{eqnarray}
\end{itembox}
特に両辺を$\mu_0$で割った式を,閉曲面Sについて面積分すると,
\begin{eqnarray*}
\frac{1}{\mu_0} \int_{S} {\rm rot} {\bf B} \cdot {\bf dS} = \int_{S} {\bf i} \cdot {\bf dS} + \varepsilon_0 \int_{S} \frac{\partial {\bf E}}{\partial t} \cdot {\bf dS}
\end{eqnarray*}
左辺をGaussの法則により体積積分に書き換えると,
\begin{eqnarray*}
\frac{1}{\mu_0} \int_{V} {\rm div}({\rm rot} {\bf B}) dV = 0
\end{eqnarray*}
右辺の1項目は閉曲面Sから湧き出る全電流を表し,2項目は閉曲面Sから湧き出る全変位電流を表す. \\
簡単のためこれらを$\alpha,\beta$と略記すると
\begin{eqnarray}
0=\alpha + \beta
\end{eqnarray}
{\bf 例}:放電により{\bf E}が弱まりつつあるコンデンサー \\
閉曲面Sを,コンデンサーの片板だけ覆うようにとる.Sから湧き出る電流は$\alpha = I(>0)$ \\
$\beta$を求める.電場{\bf E}の大きさEは極板の面積をAとすると$E=\frac{Q}{\varepsilon_0 A}$ \\
\begin{eqnarray*}
\frac{\partial E}{\partial t} = \frac{1}{\varepsilon_0 A} \frac{\partial Q}{\partial t} = - \frac{I}{\varepsilon_0 A}
\end{eqnarray*}
\begin{eqnarray*}
\beta = \varepsilon_0 \int_{S} \frac{\partial {\bf E}}{\partial t} \cdot {\bf dS} = - \varepsilon_0 \frac{I}{\varepsilon_0 A} \cdot A = -\alpha
\end{eqnarray*}
確かに(3)と一致する.

\subsection{Maxwell方程式}

\begin{itembox}[c]{Maxwell方程式}
\begin{eqnarray*}
&\textcircled{\scriptsize 1}& \quad {\rm div}{\bf E}=\frac{\rho}{\varepsilon_0} \\
&\textcircled{\scriptsize 2}& \quad {\rm div}{\bf B}=0 \\
&\textcircled{\scriptsize 3}& \quad {\rm rot}{\bf E}=-\frac{\partial {\bf B}}{\partial t} \\
&\textcircled{\scriptsize 4}& \quad {\rm rot}{\bf B}=\mu_0 \left( {\bf i} + \varepsilon_0 \frac{\partial {\bf E}}{\partial t} \right)
\end{eqnarray*}
\end{itembox}
$\textcircled{\scriptsize 1} \sim \textcircled{\scriptsize 4}$ から電荷保存が成立することを示す. \\
$\textcircled{\scriptsize 4}$の両辺を$\mu_0$で割って発散をとると
\begin{eqnarray*}
\frac{1}{\mu_0} {\rm div}({\rm rot}{\bf B}) = {\rm div}{\bf i} + \varepsilon_0 \frac{\partial}{\partial t} {\rm div}{\bf E}
\end{eqnarray*}
左辺は0となる.右辺を$\textcircled{\scriptsize 1}$を用いて書き換えると,
\begin{eqnarray*}
0 = {\rm div}{\bf i} + \frac{\partial \rho}{\partial t}
\end{eqnarray*}
確かに電荷保存が導かれる. \\
\end{document}



